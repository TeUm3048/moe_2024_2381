\section{Выводы}

В данной работе изучены свободные процессы в электрической цепи на примере RLC-контура. Особое внимание уделено связи между собственными частотами цепи и характером свободных процессов. На основе осциллограмм экспериментально определены собственные частоты и добротность контура.

В ходе эксперимента установлено, что для колебательного режима характерны комплексно-сопряженные корни, соответствующие затухающим колебаниям на осциллограммах. Также выявлены режимы апериодических процессов с действительными корнями, что подтверждает связь между расположением собственных частот на комплексной плоскости и характером процесса.

Результаты исследования для цепей различного порядка:
\begin{enumerate}
    \item Для цепи первого порядка процесс описывался экспонентой \( u(t) = A e^{-\alpha t} \), с экспериментально найденной частотой \( p_1 = -12 \) кГц, что близко к теоретической \( p_1 = -10 \) кГц.
          
    \item Для цепи второго порядка процесс описывался двумя экспонентами. Найденные частоты \( p_1, p_2 = -14 \pm 63j \) кГц близки к теоретическим \( -10 \pm 44j \) кГц. Добротность \( 2.78 \), что подтверждает колебательный процесс.
          
    \item Для цепи третьего порядка процесс описывался тремя экспонентами, с частотами \( p_1 = -10 \) кГц, \( p_2, p_3 = -25 \pm 61j \) кГц, что также соответствует теории.
\end{enumerate}

Исследование подтвердило теоретические расчеты с небольшими отклонениями в отдельных случаях.
